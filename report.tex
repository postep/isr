\documentclass[]{article}
\usepackage{polski}
\usepackage[utf8]{inputenc}
\usepackage{amsmath}
\usepackage{mathtools,leftidx}% http://ctan.org/pkg/{mathtools,leftidx}
\usepackage{graphicx}
\usepackage{float}

\usepackage[margin=0.7in]{geometry}



\title{Specyfikacja sterownika robota Velma}
\author{Jakub Postępski}
\date{7 stycznia 2019}

\begin{document}

\maketitle

\section{Przypadki użycia}

\begin{figure}[H]
	\centering
	%\includegraphics[width=0.99\linewidth]{}
	\caption{Przypadki użycia systemu}
	\label{fig:uc}
\end{figure}

\section{Struktura agentów systemu}

\begin{figure}[H]
	\centering
	%\includegraphics[width=0.99\linewidth]{}
	\caption{Struktura agentów}
	\label{fig:agents}
\end{figure}


\begin{table}[H]
	\begin{tabular}{||c|cc||}
		\hline
		Kanał & Opis & Typ \\
		\hline\hline
		Command & - & - \\
		ti & zadany prąd silnika tułowia & float \\
		te & polecenie załączenia silnika tułowia & float \\
		lt & zadane momenty stawów lewego ramienia & float[7] \\
		rt & zadane momenty stawów prawego ramienia & float[7] \\
		hpq & zadany obrót głowy & float \\
		hph & polecenie zaczęcia synchronizacji obrotu głowy & bool \\
		hpe & polecenie włączenia silnia obrotu głowy & bool \\
		htq & zadane wychylenie głowy & float \\
		hth & polecenie zaczęcia synchronizacji wychylenia  głowy & bool \\
		hte & polecenie włączenia silnia wychylenia  głowy & bool \\
		lh & polecenie obsługi lewej ręki & struktura \\
		lhr & polecenie resetowania lewej ręki & bool \\
		rh & polecenie obsługi prawej ręki & struktura \\
		rhr & polecenie resetowania prawej ręki & bool \\
		
		\hline
	\end{tabular}
	\caption{Zawartość kanałów komunikacyjnych CCVE}
\end{table}

\begin{table}[H]
	\begin{tabular}{||c|cc||}
		\hline
		Kanał & Opis & Typ \\
		\hline\hline
		Status & - & - \\
		lap & pozycja stawów lewego ramienia & float[7] \\
		rap & pozycja stawów prawego ramienia & float[7] \\
		lfts & siły i momenty odczytane z lewego czujnika siły & double[6] \\
		rfts & siły i momenty odczytane z prawego czujnika siły & double[6] \\
		lh & informacja diagnostyczna lewej ręki & struktura \\
		rh & informacja diagnostyczna prawej ręki & struktura \\
		thr & informacja o potrzebie synchronizacji stawu tułowia & bool  \\
		th & synchronizacja stawu tułowia w toku & bool \\
		te & silnik stawu tułowia aktywny & bool \\
		t & pozycja stawu tułowia & float \\
		hphr & informacja o potrzebie synchronizacji stawu wychylenia głowy & bool \\
		hph & synchronizacja stawu wychylenia głowy w toku & bool \\
		hpe & silnik stawu wychylenia głowy aktywny & bool \\
		hp & pozycja stawu wychylenia głowy & float \\
		hthr & informacja o potrzebie synchronizacji stawu wychylenia głowy & bool \\
		hth & synchronizacja wychylenia głowy w toku & bool \\
		hte & silnik stawu wychylenia głowy aktywny & bool \\
		ht & pozycja stawu wychylenia głowy & float \\
	\end{tabular}
	\caption{Zawartość kanałów komunikacyjnych CVEC}
\end{table}

\begin{table}[H]
	\begin{tabular}{||c|cc||}
		\hline
		Kanał & Opis & Typ \\
		\hline\hline
		LLWR Command & - & - \\
		llwrc & wiadomość sterowania lewego ramienia & struktura \\
		RLWR Command & - & - \\
		rlwrc & wiadomość sterowania prawego ramienia & struktura \\
		EC Command
		lfts & wiadomość sterowania lewego czujnika FTS & struktura \\
		rfts & wiadomość sterowania prawego czujnika FTS & struktura \\
		hpt & zadany obrót głowy & int \\
		hpc & konfiguracja stawu obrotu głowy & struktura \\
		htt & zadane wychylenie głowy & int \\
		htc & konfiguracja stawu wychylenia głowy & struktura \\
		tt & zadany obrót torsu & int \\
		tc & konfiguracja stawu torsu & struktura \\
		lh & wiadomość obsługi lewej ręki & struktura \\
		rh & wiadomość obsługi prawej reki & struktura \\
		\hline
	\end{tabular}
	\caption{Zawartość kanałów komunikacyjnych CVERE}
\end{table}


\begin{table}[H]
	\begin{tabular}{||c|cc||}
		\hline
		Kanał & Opis & Typ \\
		\hline\hline
		LLWR status & - & - \\
		lap & pozycja stawów lewego ramienia & float[7] \\
		rap & pozycja stawów prawego ramienia & float[7] \\
		hls & kod statusu odchylenia głowy & int \\
		lq & pozycje stawów lewego ramienia & int[7] \\
		ldq & prędkości stawów lewego ramienia & int[7]\\
		lt & momenty stawów lewego ramienia & int[7]\\
		lwr & pozycja końcówki lewego ramienia & int[6]\\
		ls & kod statusu lewego ramienia & int\\
		LLWR status & - & - \\
		rq & pozycje stawów prawego ramienia & int[7]\\
		rdq & prędkości stawów prawego ramienia & int[7]\\
		rt & momenty stawów prawego ramienia & int[7]\\
		rwr & pozycja końcówki prawego ramienia & int[6]\\
		rs & kod statusu prawego ramienia & int\\
		EC status & - & - \\
		ftsL & siły i momenty odczytane z lewego czujnika siły & double[6] \\
		ftsR & siły i momenty odczytane z prawego czujnika siły & double[6] \\
		lftss & kod statusu lewego czujnika siły & int \\
		lftsc & licznik odczytów lewego czujnika siły & int \\
		rftss & kod statusu prawego czujnika siły & int \\
		rftsc & licznik odczytów prawego czujnika siły & int \\
		hp & obrót głowy & int \\
		hpv & prędkośc  obrotu głowy & int \\
		hps & kod statusu obrotu głowy & int \\ 
		ht & odchylenie głowy & int \\
		hlv & prędkośc  odchylania głowy & int \\
		\hline
	\end{tabular}
	\caption{Zawartość kanałów komunikacyjnych CREVE}
\end{table}

\section{Specyfikacja podystemu sterowania}

\begin{table}[H]
	\begin{figure}[H]
		\centering
		%\includegraphics[width=0.99\linewidth]{}
		\caption{Automat skończony podsystemu sterowania}
		\label{fig:agents}
	\end{figure}

	
	\begin{table}[H]
		\begin{tabular}{||c|c||}
		Stan & Zachowanie \\
		\hline \hline

			
		\end{tabular}
		\caption{Przyporządkowanie zachowań do kolejnych stanów podsystemu sterowania}
	\end{table}

	\begin{table}[H]
		\begin{tabular}{||cc|c||}
		Stan początkowy & Stan końcowy & Zachowanie \\
		\hline \hline
		safe & cart\_imp & $\neg{IN\_ERROR} \and recvCartImpCmd $ \\
		safe & jnt\_imp & $\neg{IN\_ERROR} \and recvJntImpCmd $ \\
		safe & safe\_col & $\neg{IN\_ERROR} \and recvSafeColCmd $ \\
		safe\_col & cart\_imp $\neg{IN\_ERROR} \and recvCartImpCmd $ \\
		safe\_col & jnt\_imp $\neg{IN\_ERROR} \and recvJntImpCmd $ \\
		\end{tabular}
		\caption{Warunki przejść pomiędzy stanami podsytemu sterowania}
	\end{table}

	\begin{tabular}{||c|cc||}
		
	\end{tabular}
	\caption{Kompozycja funkcji przejść}
\end{table}

\end{document}
\documentclass[]{article}
\usepackage{polski}
\usepackage[utf8]{inputenc}
\usepackage{amsmath}
\usepackage{mathtools,leftidx}% http://ctan.org/pkg/{mathtools,leftidx}
\usepackage{graphicx}
\usepackage{float}

\usepackage[margin=0.7in]{geometry}



\title{Specyfikacja sterownika robota Velma}
\author{Jakub Postępski}
\date{7 stycznia 2019}

\begin{document}

\maketitle

\section{Przypadki użycia}

\begin{figure}[H]
	\centering
	\includegraphics[width=0.99\linewidth]{}
	\caption{Przypadki użycia systemu}
	\label{fig:uc}
\end{figure}

\section{Struktura agentów systemu}

\begin{figure}[H]
	\centering
	\includegraphics[width=0.99\linewidth]{}
	\caption{Struktura agentów}
	\label{fig:agents}
\end{figure}


\begin{table}[H]
	\begin{tabular}{||c|cc||}
		\hline
		Kanał & Opis & Typ \\
		\hline\hline
		Command & - & - \\
		cip & Polecenie ruchu w przestrzeni kartezjańskiej & struktura \\
  		jip & Polecenie ruchu w przestrzeni stawów & struktura \\
  		scl & Polecenie ruch w trybie Self Collision Avoidance & bool \\
		si & polecenie rozpoczęcia inicjalizacji & bool \\
		ei & polecenie zakończenia inicjalizacji & bool \\
		es & polecenie awaryjnego zatrzymania & bool \\
		\hline
	\end{tabular}
	\caption{Zawartość kanałów komunikacyjnych CCVE}
\end{table}

\begin{table}[H]
	\begin{tabular}{||c|cc||}
		\hline
		Kanał & Opis & Typ \\
		\hline\hline
		Command & - & - \\
		cip & Polecenie ruchu w przestrzeni kartezjańskiej & struktura \\
		jip & Polecenie ruchu w przestrzeni stawów & struktura \\
		scl & Polecenie ruch w trybie Self Collision Avoidance & bool \\
		si & polecenie rozpoczęcia inicjalizacji & bool \\
		ei & polecenie zakończenia inicjalizacji & bool \\
		es & polecenie awaryjnego zatrzymania & bool \\
		\hline
	\end{tabular}
	\caption{Zawartość kanałów komunikacyjnych CVEC}
\end{table}

\begin{table}[H]
	\begin{tabular}{||c|cc||}
		
	\end{tabular}
	\caption{Zawartość kanałów komunikacyjnych CVERE}
\end{table}


\begin{table}[H]
	\begin{tabular}{||c|cc||}
		
	\end{tabular}
	\caption{Zawartość kanałów komunikacyjnych CREVE}
\end{table}

\section{Specyfikacja podystemu sterowania}

\begin{table}[H]
	\begin{figure}[H]
		\centering
		\includegraphics[width=0.99\linewidth]{}
		\caption{Automat skończony podsystemu sterowania}
		\label{fig:agents}
	\end{figure}

	
	\begin{table}[H]
		\begin{tabular}{||c|cc||}
			
		\end{tabular}
		\caption{Przyporządkowanie zachowań do kolejnych stanów podsystemu sterowania}
	\end{table}

	\begin{table}[H]
		\begin{tabular}{||c|cc||}
			
		\end{tabular}
		\caption{Warunki końcowe funkcji przejścia kolejnych zachowań}
	\end{table}

	\begin{tabular}{||c|cc||}
		
	\end{tabular}
	\caption{Kompozycja funkcji przejść}
\end{table}

\end{document}